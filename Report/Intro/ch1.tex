Respiratory Syncytial Virus, also known as RSV, is a common virus that affects the respiratory system without inducing severe symptoms in most cases. RSV is spread through transmission of respiratory secretions when in close contact with infected individuals and the severity of symptoms varies each year. Young children and older adults are, respectively, the first and second highest risk groups to develop a harsher outcome making additional studies necessary. Two vaccines are currently available: Arexvy (GSK) and Abrysvo (Pfizer) of which the latter can be administered to pregnant women as well. Another option for immunization consists in the injection of the RSV antibody especially for infants and toddlers.\par
As the name itself suggests, the virus is able to form syncytia through cell-to-cell fusion which allows viruses to reach other cells and to evade the host immune system \cite{gower}. Syncytia seems to be created thanks to viral envelope proteins which work together with host proteins that maintain membrane integrity, adhesion and cell mobility. One study suggested that RhoA, a host small GTPase, plays a role in the formation of syncytia, but further investigations are needed to define the exact step at which the protein directly acts \cite{gower}.\par
Each year 15,000 to 20,000 cases of RSV have been reported in Spain with 400 out of 100,000 children younger than a year hospitalized with consequently high costs for the Spanish public healthcare system \cite{rsv}.\par
Some scientists have found possible implications of Covid-19 on RSV diffusion. An increase in off-season cases of RSV infections, as well as a shift towards older children, was noticed during the first year of SARS-CoV-2 pandemics firstly in Australia \cite{raya}. A possible hypothesis lies in the induced immune dysregulation caused by Covid-19 with downregulation of CD19 in B-cells \cite{jing}.\par
Mathematical models are a significant tool to improve our knowledge about how diseases behave in time and should provide a better understanding of how they spread among individuals. Moreover they acquire a major importance when realizing that not only usual environmental factors, but other diseases as well (e.g. Covid-19) might play crucial roles in RSV transmission.\par
Here, the main aim is to determine how external factors influence the overall dynamics of RSV diffusion taking into account stochastic terms as well to better mimic environmental fluctuations.\par
The analyzed paper deals with the model proposed in Weber et al. \cite{weber} to determine the effects of perturbations on specific parameters of the RSV model. RSV varies in timing and severity each year suggesting that several factors such as temperature, humidity, pollution and others might play a role in shaping RSV dynamics \cite{weber}.