Deterministic and stochastic approaches can serve different purposes in understanding how biological system works through mathematical models. Here, stochastic models proved to better interpolate the intrinsic variability of data and environmental factors compared to deterministic models which might be used when the environment does not substantially influence the simulation.\par
The authors tries two different perturbation ranges (100\% and 5\%), but do not provide any information on the reasons that led them to choose those specific values. They also state that small perturbations, e.g. 2\%, result in a drastic decrease of the infected population, showing that even small changes can limit the spread of RSV. For these reasons we suggest a sensitivity analysis to better determine the role of these perturbations on the simulation, a fundamental aspect for governments to apply effective healthcare policies.\par
While performing the stochastic simulation we also noticed that a value for parameter $\alpha$ is given only for the version with perturbation on the birth rate, whereas the one for the perturbation on the transmission rate is missing. In this last case, the only information provided is the perturbation range of 5\% without a specific value. Therefore, we hypothesize that $\alpha$ represents a fraction of the parameter, defining fluctuations above and below the reference parameter value by the specified alpha factor.\par
Since the transmission rate seems to be highly sensitive to perturbations, we suggest that health policies should act on it to ensure an effective management of epidemics. Moreover, the increasing amount of people moving across different countries and the exacerbation of meteorological conditions in the last years might seriously affect the spread of disease, making modelling studies essential.